\documentclass{article}
\usepackage[utf8]{inputenc}
\usepackage[shortlabels]{enumitem} %needed for lettered lists
\usepackage{amssymb}
\usepackage{amsthm}
\usepackage{amsmath}
\usepackage{tikz}  %%%% Needed for fsa diagrams

% the preamble
\linespread{1.3}    %%%%%%%% This increases space between lines
\title{Assignment 6}
\author{Ayman Shahriar, UCID: 10180260}
\date{\today}

% start of the document content
\begin{document}
% command to display the title
\maketitle

%Question1
\textbf{1)} Let $L = \{a^i b^j c^{i+j} : i, j \ge 0 \}$. Create a grammar $G$ and prove that $L(G) = L$.\\
Context Free Grammar $G$ is:

$S \rightarrow aSc | T$

$T \rightarrow bTc | \epsilon$\\ 
Before we prove that $L(G) = L$, it will be useful to prove by induction on $i$ that $S \Rightarrow^\star a^iTc^i$ for all $i \ge 0$
\begin{proof}
Base Case (i = 0): Since we have the production $S \rightarrow T$, we can derive $S \Rightarrow^\star T$, where $T = a^0Tc^0$.\\
Inductive Step: Suppose $k \ge 0$ is an integer such that $S \Rightarrow^\star a^kTc^k$ (Inductive Hypothesis).\\
We need to show that $S \Rightarrow^\star a^{k+1}Tc^{k+1}$.\\
Using the inductive hypothesis and the production $S \rightarrow aSc$ in our grammar, we can derive $S \Rightarrow aSc \Rightarrow^\star aa^kTc^kc$\\
So $S \Rightarrow^\star a^{k+1}Tc^{k+1}$, as required.\\
Conclusion: Thus, $S \Rightarrow^\star a^iTc^i$ for all $i \ge 0$
\end{proof} 
Also, before proving $L(G) = L$, it will be useful to consider the production $T \rightarrow bTc | \epsilon $ to be the grammar $G_T$.\\
\medskip
Then $L(G_T) = \{b^ic^i: i \ge 0\}$.\\
Now we will prove that $L(G) = L$ by first proving that $L(G) \subseteq L$ and then proving that $L \subseteq L(G)$
\begin{proof}
Suppose $w \in L(G)$. Then $S \Rightarrow^\star w$\\
Note that for any derivation of a word in $L(G)$, initially the production  $S \rightarrow aSc$ is used 0 or more times until the production $S \rightarrow T$ is used (and then the productions  $T \rightarrow bTc|\epsilon$ are used to finish deriving the word).\\
This means that in the derivation of any word in $x \in L(G)$, there exists a sentinel form $a^iTc^i, i \ge 0$ such that $S \Rightarrow^\star a^iTc^i \Rightarrow^\star x$\\
Then since $S \Rightarrow^\star w$, we get $S \Rightarrow^\star a^iTc^i \Rightarrow^\star w$ for some $i \ge 0$.\\
Now we can rewrite $w = a^iyc^i$, where $T \Rightarrow^\star y$\\
Since $T$ is the seed variable of $G_T$ (the grammar that we defined before the proof), it means that $y \in L(G_T)$, so $y = b^jc^j$ for some $j \ge 0$.\\
Then $w = a^ib^jc^jc^i = a^ib^jc^{i+j} \in L$\\
Thus, $L(G) \subseteq L$
\end{proof}
Now we will prove that $L \subseteq L(G)$
\begin{proof}
Suppose $w \in L$. Then $w = a^ib^jc^{i+j}$ for some $i,j \ge 0$\\
(We need to show that $S \Rightarrow^\star w$)\\
Now, we know that $S \Rightarrow^\star a^nTc^n$ for all $n \ge 0$\\
Then $S \Rightarrow^\star a^iTc^i$\\
Since $b^jc^j \in L(G_T)$, it means that $T \Rightarrow^\star b^jc^j$\\
So $S \Rightarrow^\star a^iTc^i \Rightarrow^\star a^ib^jc^jc^i$\\
Then since $w = a^ib^jc^jc^i = a^ib^jc^{i+j}$, it means that $S \Rightarrow^\star w$\\
Thus, $L \subseteq L(G)$
\end{proof}

Since we have proved that $L(G) \subseteq L$ and $L \subseteq L(G)$, it means that $L(G) = L$\\

\textbf{2a)} Let $L_1 = \{w \in \{a, b\}^\star : w = w^r\}$. This is the language of palindromes over $\{a, b\}$, and from class we know that the the language of palindromes are always non-regular.\\
\medskip
Since $L_1$ can be generated by context-free grammars, $L_1$ is also context-free.\\
For example, a context-free grammar that generates $L_1$ is:

$S \rightarrow \epsilon$

$S \rightarrow a$

$S \rightarrow b$

$S \rightarrow aSa$

$S \rightarrow bSb$\\ \\
Note that $prefix(L_1) = \{w \in \Sigma^\star\}$
\begin{proof}
Suppose $x \in \Sigma^\star$.\\
Then $x$ is a prefix of the palindrome $xx^r$.\\
Thus, all words over $\Sigma$ are prefices of palindromes, so $prefix(L_1) = \{w \in \Sigma^\star\}$
\end{proof}
Also, $prefix(L_1)$ can be generated by the regular expression $(a+b)\star$, so that means $prefix(L_1)$ is regular.\\
So $L_1$ is a non-regular, context free language and $prefix(L_1)$ is regular.\\ \\
\textbf{2b)} Let $L_2 = \{a^ib^i : i \ge 0\}$. In class we have shown that this language is non-regular.\\
This language is context-free because it can be generated by context free grammars\\
A grammar that generates $L_2$ is:

$S \rightarrow aSb|\epsilon$\\ 
Note that $prefix(L_2) = \{a^ib^j: i \ge j \}$
\begin{proof}
Suppose $w \in L_2$. Then $w = a^ib^i$ for some $i \ge 0$\\
We can rewrite $w$ as $w = xy$ such that $x$ is a prefix of $w$ and $x, y \in \Sigma^\star$\\
Then we have two cases:\\
Case 1: $x$ does not contain any $b's$. Then $x = a^mb^0$ and $y = a^{i-m}b^i$ for some $m \ge 0$\\
Case 2: $x$ contains both $a's$ and $b's$. Then $y$ cannot contain any $a's$, so $x = a^ib^{i-n}$ and $y = b^n$ for some $n \ge 0$\\
So in both cases, the prefix $x = a^mb^n$ such that $m \ge n$ and $m, n \ge 0$\\
So $prefix(L_2) = \{a^ib^j: i \ge j \}$
\end{proof}
Now we will prove that $prefix(L_2)$ is non-regular using the pumping lemma for regular languages and contradiction.
\begin{proof}
Suppose $prefix(L_2)$ is regular. Then it has a pumping length $n \ge 0$.\\
Let $w = a^nb^n$. Then $w \in prefix(L_2)$ because $n \ge n$, and note that $|w| = |a^nb^n| = 2n \ge n$\\
Now suppose $w = xyz$ such that $|xy| \le n$ and $|y| \ge 1$ (so $y \neq \epsilon$) \\
Then since $w = a^nb^n$ and $|xy| \le n$, we must have that $xy = a^ia^j$ for some $0 < i+j \le n$ (that is, $xy$ can only consist of $a's$).\\
Let $x = a^i$ and $y = a^j$ (where $i \ge 0$ and $j \ge 1$ since $y \neq \epsilon$).\\
Then $z = a^{n-i-j}b^n$.\\
Now, consider the word $w^{(0)} = xy^0z$\\
This is equal to $a^ia^{n-i-j}b^n = a^{n-j}b^n$\\
Since $j \ge 1$, then $n-j < n$, so $w^{(0)}$ is not an element of $prefix(L_2)$.\\
So the pumping lemma does not hold for this language, which contradicts the assumption that $prefix(L_2)$ is regular.\\
Thus, $prefix(L_2)$ is not regular.
\end{proof}

So we have proved that $L_2$ is a non-regular, context free language and $prefix(L_2)$ is non-regular.

















\end{document}