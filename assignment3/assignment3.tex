\documentclass{article}
\usepackage[utf8]{inputenc}
\usepackage[shortlabels]{enumitem} %needed for lettered lists
\usepackage{amssymb}
\usepackage{amsthm}
\usepackage{amsmath}

% the preamble
\title{Assignment 2}
\author{Ayman Shahriar, UCID: 10180260}
\date{\today}

% start of the document content
\begin{document}
% command to display the title
\maketitle

% Question 1
\textbf{1)} Suppose $M = (Q, \Sigma, \delta, q_0, A) $ and let $\delta^\star: Q \times \Sigma \rightarrow Q$ be it's extended transition function.\\
Suppose $q \in Q$ and $x, y \in \Sigma^\star$\\
We will use induction on $|x| = n$ to prove that $\delta^\star(q, xy) = \delta^\star(\delta^\star(q, x), y)$\\ \\
But first, it will be useful to prove the following property:\\ 
$\delta^\star(q, ay) = \delta^\star(\delta(q, a), y)$ for any $q \in Q$, $a \in \Sigma$ and $y \in \Sigma^\star$ \\
We will prove this by using induction on $|y| = n$
\begin{proof}
Basis (n = 0): If $|y| = 0$, then $y = \epsilon$\\
Then $\delta^\star(q, ay) = \delta^\star(q, a \epsilon) = \delta^\star(\delta(q, a), \epsilon) = \delta^\star(\delta(q, a), y)$, as required for this case\\ \\
Inductive step: Suppose $k \ge 0$ is an integer such that $\delta^\star(q, ay) = \delta^\star(\delta(q, a), y)$, where $|y| = k$ (Inductive Hypothesis)\\
We want to prove that $\delta^\star(q, ay) = \delta^\star(\delta(q, a), y)$ for $|y| = k+1$\\
Now, suppose $|y| = k+1$, and $y = mb$ where $b \in \Sigma, m \in \Sigma^\star$\\
Then $\delta^\star(q, ay) = \delta^\star(q, amb)$\\
$= \delta(\delta^\star(q, am), b)$ (by definition of $\delta$)\\
$= \delta(\delta^\star(\delta(q, a), m), b)$ (by using the inductive hypothesis, since $|m| = |y|-1 = k$)\\
$= \delta^\star(\delta(q, a), mb)$ (by definition of $\delta$)\\
$= \delta^\star(\delta(q, a), y)$, as required\\
Thus, $\delta^\star(q, ay) = \delta^\star(\delta(q, a), y)$ for any $q \in Q$, $a \in \Sigma$ and $y \in \Sigma^\star$ \\
\end{proof}
Now we will move on to our main proof:
\begin{proof}
Basis (n = 0): Then $|x| = 0$, so $x = \epsilon$\\
Then $\delta^\star(q, x) = \delta^\star(q, \epsilon ) = q$\\
So $\delta^\star(q, xy) = \delta^\star(q, \epsilon y) = \delta^\star(q, y) = \delta^\star(\delta^\star(q, x), y)$, as required for this case\\ \\
Inductive step: Suppose $k \ge 0$ is an integer such that $\delta^\star(q, xy) = \delta^\star(\delta^\star(q, x), y)$, where $|x| = k$\\
We want to prove that $\delta^\star(q, xy) = \delta^\star(\delta^\star(q, x), y)$ for $|x| = k+1$\\
Let $x = ma$ where $|x| = k+1$, $m \in \Sigma^\star$, $|m| = k$ and $a \in \Sigma$\\
Then $\delta^\star(q, xy) = \delta^\star(q, may)$\\
$= \delta^\star(\delta^\star(q, m), ay)$ \ (by the inductive hypothesis, since $|m| = |x|-1 = k$)\\
$= \delta^\star(\delta(\delta^\star(q, m), a), y)$ \ (by using the property proved before)\\
$ = \delta^\star(\delta^\star(q, ma), y)$ \ (by definition of $\delta^\star$)\\
$= \delta^\star(\delta^\star(q, x), y)$, as required\\ \\
Thus, $\delta^\star(q, xy) = \delta^\star(\delta^\star(q, x), y)$
for any $q \in Q$ and $x, y \in \Sigma^\star$\\
\end{proof}

% Question 2
\textbf{2)} Suppose $M = (Q, \Sigma, \delta, q_0, A)$ is a deterministic finite state machine such that $L(M) = L$. Suppose that $|Q| = 2$. Let $Q = \{a, b\}$. We have the following cases:\\
\textbf{case 1:} $A = \emptyset$\\
Then $\forall x \in \Sigma^\star$, $\delta^\star(q_0, x) \notin A$\\
So $L = \emptyset$\\ \\
\textbf{case 2:} $q_0 = q_a, A = \{q_a\}$ \\
Then $q_0 \in A$, so $\delta^\star(q_0, \epsilon) \in A$\\
So $\epsilon \in L(M) = L$\\ \\
\textbf{case 3:} $ q_0 = q_b, A = \{q_b\}$\\
Then $q_0 \in A$, so $\delta^\star(q_0, \epsilon) \in A$\\
So $\epsilon \in L(M) = L$\\ \\
\textbf{case 4:} $q_0 = q_a, A = \{q_a, q_b\}$\\
Then $q_0 \in A$, so $\delta^\star(q_0, \epsilon) \in A$\\
So $\epsilon \in L(M) = L$\\ \\
\textbf{case 5:} $q_0 = q_b, A = \{q_a, q_b\}$\\
Then $q_0 \in A$, so $\delta^\star(q_0, \epsilon) \in A$\\
So $\epsilon \in L(M) = L$\\ \\
\textbf{case 6:} $q_0 = q_a, A = \{q_b\}$\\
If $\delta(q_a, a) = q_b$ for some $a \in \Sigma$, then\\
$\delta(q_a, a) = q_b \Leftrightarrow \delta(\delta^\star(q_0, \epsilon), a) = q_b$\\
$\Leftrightarrow \delta^\star(q_a, a) = q_b$\\
$\Leftrightarrow \delta^\star(q_a, a) \in A$ (since $q_b$ is the only element of $A$)\\ 
$\Leftrightarrow a \in L(M) = L$\\ \\
If $\delta(q_a, a) \neq q_b$ for all $a \in \Sigma$,\\ 
then $\delta(q_a, a) = q_a$ for all $a \in \Sigma$\\
So $\forall x \in \Sigma^\star, \delta^\star(q_0, x) \neq q_b$ (where $q_0 = q_a$)\\
Then $\delta^\star(q_0, x) \notin A$\\
So $\forall x \in \Sigma^\star, x \notin L$\\
So $L = \emptyset$\\ \\ \\
\textbf{case 7:} $q_0 = q_b, A = \{q_a\}$\\
If $\delta(q_b, a) = q_a$ for some $a \in \Sigma$, then\\
$\delta(q_b, a) = q_a \Leftrightarrow \delta(\delta^\star(q_0, \epsilon), a) = q_a$\\
$\Leftrightarrow \delta^\star(q_b, a) = q_a$\\
$\Leftrightarrow \delta^\star(q_b, a) \in A$ (since $q_a$ is the only element of $A$)\\
$\Leftrightarrow a \in L(M) = L$\\ \\
If $\delta(q_b, a) \neq q_a$ for all $a \in \Sigma$,\\ 
then $\delta(q_b, a) = q_a$ for all $a \in \Sigma$\\
So $\forall x \in \Sigma^\star, \delta^\star(q_0, x) \neq q_a$ (where $q_0 = q_b$)\\
Then $\delta^\star(q_0, x) \notin A$\\
So $\forall x \in \Sigma^\star, x \notin L$\\
So $L = \emptyset$ \\ \\
Thus, for all cases where $|Q| = 2$, $L = \emptyset$ or $\epsilon \in L$ or $\exists a \in \Sigma$ such that $a \in L$. \\ \\


%Queston 3
\textbf{3)} Suppose $L$ is a finite state language over the aplhabet $\Sigma$. Then $L$ is regular, so there exists a finite state machine $M = \{Q, \Sigma, \delta, q_0, A \}$ such that $L(M) = L$. Since $L$ is finite, there exists a word whose length is greater than or equal to the length of all the other words in $L$. Suppose the length of the longest word in $L$ is $n$.\\ 
We will define each component of M:\\ \\
$Q = \{q_x: x \in \Sigma^\star, |x| \le n \} \cup \{q_{ds}\} $ \\
Each state $q_x \in Q$ corresponds to a word $x \in \Sigma^\star$ whose length is less than or equal to the longest word in L. Also, there is a dead state $q_{ds}$ that will loop back to itself whatever the input.\\ \\
$\Sigma$ = the alphabet that $L$ uses\\ \\
Let $a \in \Sigma$\\
If $|x| < n$, then $\delta(q_x, a) = q_y$, where $y = xa$ and $|y| = |x| + 1$\\
If $|x| = n$, then $\delta(q_x, a) = q_{ds}$\\ \\
$q_0 = q_x$, where $|x| = 0$. Then $x = \epsilon$, So $q_0 = q_{\epsilon}$\\ \\
$A = \{q_x \in Q: x \in L\}$\\ \\
Proof that $L(M) = L$:\\
\begin{proof} Suppose $x \in L(M)$. \\ 
$\Leftrightarrow \delta^\star(q_0, x) \in A$ (by definition of a machine accepting a word)\\ 
Now we have two cases:\\ \\
Case 1: $x \neq \epsilon$\\
Then $\delta^\star(q_0, x) \in A \Leftrightarrow \delta^\star(q_0, ay) \in A$ (where $x = ay, a \in \Sigma, y \in \Sigma^\star, |y| = |x|-1$)\\
$\Leftrightarrow \delta(\delta^\star(q_0, a), y) \in A$ (by definition of $\delta^\star)$\\
$\Leftrightarrow \delta(\delta^\star(q_0, \epsilon a), y) \in A$\\
$\Leftrightarrow \delta(\delta(\delta^\star(q_0, \epsilon), a), y) \in A$ (by definition of $\delta^\star$)\\
$\Leftrightarrow \delta(\delta(q_0, a), y) \in A$ (by definition of $\delta^\star)$\\
$\Leftrightarrow \delta(\delta(q_\epsilon, a), y) \in A$ (since $q_0 = q_\epsilon$)\\
$\Leftrightarrow \delta(q_{\epsilon a}, y) \in A$ (by definition of $\delta$, where $|\epsilon| < |x|$ so $|\epsilon| < n$ ) \\
$\Leftrightarrow \delta(q_{a}, y) \in A$ \\
$\Leftrightarrow q_{ay} \in A$ (by definition of $\delta$ ,where $|y| = |x|-1$ so $|y| < n)$ \\
$\Leftrightarrow q_x \in A$ \\
Now, since $A = \{q_w \in Q: w \in L\}$, and $q_x \in A$, we get\\ 
$q_x \in A \Leftrightarrow x \in L$, as required for this case\\ \\
Case 2: $x = \epsilon$\\
Then $\delta^\star(q_0, x) \in A \Leftrightarrow \delta^\star(q_0, \epsilon) \in A$\\
$\Leftrightarrow q_0 \in A$ (by definition of $\delta^\star$)\\
$\Leftrightarrow q_\epsilon \in A$ (since $q_0 = q_\epsilon$)\\
Now, since $A = \{q_w \in Q: w \in L\}$, and $q_\epsilon \in A$, we get\\
$q_x \in A \Leftrightarrow x \in L$, as required for this case\\ \\
Thus, we have proved that $L(M) = L$

\end{proof}

\end{document}